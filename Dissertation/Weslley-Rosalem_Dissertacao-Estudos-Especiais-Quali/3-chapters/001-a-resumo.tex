%\chapter{Resumo}
%\label{cap:resumo}
\begin{resumo}

Com a crescente transformação digital, gerenciar ambientes de TI que são ao mesmo tempo complexos e dinâmicos, tornou-se um desafio cada vez maior. A inteligência Artificial em Operações de TI ou (AIOps), surge como uma solução que integra aprendizado de máquina e \textit{big data} para automatizar tarefas de gerenciamento de TI, tais como detecção de anomalias, previsão de capacidade, correlação de eventos e identificação de causas raízes. O objetivo principal deste estudo é combinar a arquitetura \textit{Transformer}, com dados de séries temporais provenientes do \textit{Prometheus}, sistema de monitoramento e banco de dados de séries temporais, a fim de melhorar a análise preditiva e a detecção de anomalias em sistemas de TI. A pesquisa realiza uma revisão sistemática da literatura, fundamentando as bases teóricas das tecnologias discutidas. A seção metodológica detalha os conjuntos de dados utilizados, as etapas de pré-processamento e as técnicas de análise preditiva adotadas. Os resultados deste estudo podem contribuir significativamente para a eficiência das operações de TI, facilitando o gerenciamento das complexidades presentes nos ambientes atuais.

\textbf{Palavras-chave:} \textit{Deep Learning; Time Series; Transformer; AIOPs}.

\end{resumo}
