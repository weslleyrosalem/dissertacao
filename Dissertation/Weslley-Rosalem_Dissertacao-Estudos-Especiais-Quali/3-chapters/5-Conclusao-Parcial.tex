\chapter{Conclusão Parcial} \label{Cap_conclusao}

Nesta fase da pesquisa, observa-se que os resultados preliminares obtidos com a arquitetura \textit{Transformer} aplicada às séries temporais do \textit{Prometheus} são promissores. A capacidade do modelo de capturar dependências de longo prazo nas séries temporais demonstra sua aplicabilidade potencial em cenários de AIOps, especificamente para monitoramento e previsão de métricas de sistemas de TI.

Esta seção serve como um roteiro para essas futuras investigações, abordando desde o ajuste fino de hiperparâmetros até estratégias mais sofisticadas para avaliar a complexidade do modelo. Também vale ressaltar a importância da validação cruzada, como uma ferramenta para assegurar a robustez e a generalização do modelo em cenários de TI diversos e dinâmicos. 

O desempenho do modelo, apesar de ainda não ser o desejado, representa um passo adiante na aplicação de \textit{Transformers} em séries temporais. O emprego de técnicas de otimização de hiperparâmetros e ajuste fino poderá, possivelmente, elevar este modelo ao estado da arte em previsão de séries temporais.

Embora os resultados sejam encorajadores, é importante reconhecer as limitações inerentes ao estudo. Primeiramente, a dependência da qualidade e da quantidade de dados disponíveis é um fator crucial que influencia o desempenho do modelo. Além disso, a complexidade da arquitetura \textit{Transformer}, exige um balanceamento cuidadoso dos hiperparâmetros para evitar o sobreajuste e garantir a generalização do modelo.

As implicações práticas desses resultados são significativas. A capacidade de prever anomalias em sistemas de TI com antecedência pode permitir uma resposta mais rápida a possíveis problemas, minimizando o tempo de inatividade e melhorando a eficiência operacional. No entanto, é fundamental continuar a pesquisa para aprimorar a precisão e a confiabilidade do modelo em diferentes conjuntos de dados e cenários.

Para continuação e trabalhos futuros, propõe-se uma abordagem mais granular na sintonização dos \textit{hyperparameters}, tal tarefa será realizada através algoritmos de otimização baseado em  meta-heurística, como PSO, \textit{Manta-Ray}, entre outros,  ou até mesmo de forma mais simples com \textit{Bayesian optimization}. Este ajuste fino é um passo crítico para maximizar a eficiência do modelo. Ademais, propõe-se também a exploração de diferentes abordagens de pré-processamento de dados, bem como a experimentação com outras arquiteturas de redes neurais.

Em resumo, este estudo representa um passo importante na aplicação de técnicas avançadas de aprendizado de máquina para a otimização de operações de TI. Continuar a pesquisa nessa direção não só contribuirá para o campo acadêmico, mas também oferecerá soluções práticas para desafios do mundo real no gerenciamento de sistemas de TI.



\section{Cronograma}
O cronograma ilustrado na tabela \ref{tab:cronograma} delineia o plano de ação para as fases restantes, desde a revisão e correções sugeridas pela banca até a defesa final da dissertação. 

\begin{table}[h]
\begin{tabular}{|c|c|c|c|c|c|c|c|c|c|c|c|}
\hline
 & Ago & Set & Out & Nov & Dez & Jan & Fev & Mar & Abril & Maio & Jun \\ \hline
\begin{tabular}[c]{@{}c@{}}Correções \\ Pontuadas\\ pela Banca\end{tabular} & X & X &  &  &  &  &  &  &  &  &  \\ \hline
\begin{tabular}[c]{@{}c@{}}Metodologia / \\ Experimentos\end{tabular} & X & X & X &  &  &  &  &  &  &  &  \\ \hline
Desenvolvimento & X & X & X & X &  X&  &  &  &  &  &  \\ \hline
Qualificação &  &  &  &  & X &  &  &  &  &  &  \\ \hline
\begin{tabular}[c]{@{}c@{}}Escrita e Submissão\\ de Artigos\end{tabular} &  &  &  &  &  & X & X & X & X & X &  \\ \hline
Defesa &  &  &  &  &  &  &  &  &  &  & X \\ \hline
\end{tabular}
\caption{Cronograma de Trabalho até a Defesa}
\label{tab:cronograma}
\end{table}




