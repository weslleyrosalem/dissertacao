\chapter{INTRODUÇÃO} \label{cap_introducao}

Em meio à era da transformação digital, os ambientes de Tecnologia da Informação (TI), evoluíram para atender diversos cenários críticos, de diferentes setores, incluindo mas não limitando-se a: saúde, financeiro, telecomunicações, segurança, logística, entre outros. Visando melhorar a disponibilidade, resiliência,  flexibilidade e confiabilidade, eles também tornaram-se notavelmente mais complexos e dinâmicos. A crescente dependência das empresas em infraestruturas de TI para conduzir suas operações diárias, tem impulsionado a implementação de sistemas distribuídos em larga escala que são intrinsecamente desafiadores de monitorar, escalar, manter e prever a quantidade de recursos necessários \cite{pahl2015containerization}.

O gerenciamento desses vastos ambientes de TI é desafiador, dada a sua natureza heterogênea e a multiplicidade de componentes de hardware e software que integram. Os ambientes de TI contemporâneos englobam uma variedade de sistemas, desde servidores físicos, máquinas virtuais, contêineres, redes, bancos de dados e aplicações \cite{pahl2015containerization}. A adoção crescente da computação em nuvem amplia ainda mais essa complexidade, com recursos de TI distribuídos geograficamente.

Nesse cenário, as operações de TI são pressionadas a garantir a disponibilidade e o ótimo desempenho  desses sistemas. Uma das tarefas mais árduas é a identificação de causa raiz, que busca determinar a origem de falhas ou problemas em um sistema. Esta tarefa é exacerbada pela interdependência dos componentes e pela vastidão de dados gerados \cite{cohen2004correlating}.

A manutenção por sua vez, apresenta seus próprios desafios. Assegurar sistemas atualizados e protegidos contra vulnerabilidades, demanda monitoramento contínuo e frequentemente, a coordenação de atualizações em múltiplos sistemas. O processo de manutenção precisa ser meticuloso para minimizar interrupções em serviços  críticos.

Diante desses desafios, as organizações buscam soluções que possam automatizar e otimizar suas operações. A Inteligência Artificial para Operações de TI, conhecida como \textit{Artificial Inteligence for I.T Operations} (AIOps), emerge como uma disciplina que emprega técnicas de IA para abordar problemas operacionais. AIOps integra \textit{big data}\footnote{\textit{Big Data} refere-se a conjuntos de dados que são tão grandes e complexos que exigem métodos avançados e ferramentas de análise para processar. Estes dados podem ser estruturados ou não estruturados e são caracterizados por volume, velocidade e variedade.}
 e aprendizado de máquina, visando aprimorar aspectos como detecção de anomalias, previsão de capacidade, correlação de eventos e identificação de causas raízes \cite{sill2019aiops}.

No âmbito do AIOps, destaca-se o uso de aprendizado profundo\footnote{Aprendizado profundo, ou \textit{deep learning}, é uma subárea da aprendizagem de máquina \textit{(machine learning)}, que por sua vez é um domínio da inteligência artificial (IA). Essa técnica se inspira na estrutura e funcionamento do cérebro humano, mais especificamente nas redes neurais biológicas, para processar dados e criar padrões para a tomada de decisões.} para análise de séries temporais\footnote{Uma série temporal é uma sequência de pontos de dados coletados ou registrados em intervalos de tempo regulares ou irregulares. Cada ponto de dados na série temporal é um registro de alguma atividade, observação ou medição em um momento específico. Tecnicamente, uma série temporal é uma sequência de observações  x(t), onde  (t) representa o tempo em que a observação foi feita.}, como métricas de desempenho (KPI). A arquitetura \textit{Transformer}, originalmente concebida para processamento de linguagem natural, tem se mostrado eficaz em modelar complexidades em séries temporais, capturando dependências de longo alcance \cite{vaswani2017attention}.

Ademais, o \textit{Prometheus}, um sistema de monitoramento e banco de dados de séries temporais, é amplamente adotado para coletar métricas de sistemas de TI. Ao integrar arquitetura \textit{Transformer} com métricas do \textit{Prometheus}, é possível desenvolver modelos de aprendizado de máquina mais robustos para análise preditiva e detecção de anomalias.

Neste trabalho, será explorado a aplicação de \textit{Transformer} em dados de séries temporais obtidos via \textit{Prometheus}, visando identificar anomalias e prever tendências de consumo de recursos. Esta abordagem tem o potencial de aprimorar substancialmente a eficiência das operações de TI, facilitando o gerenciamento da crescente complexidade dos ambientes atuais. Nos capítulos subsequentes, serão abordados os fundamentos teóricos das tecnologias em foco, a metodologia adotada e os resultados alcançados, bem como a estrutura organizacional da pesquisa, sendo:

\begin{itemize}
    \item (Capítulo \ref{cap_introducao}), em que é apresentado o tema objeto da pesquisa de forma com que sejam justificados os esforços na realização da presente dissertação, bem como realizada a contextualização, apresentados os objetivos geral e específicos e a metodologia empregada;
    \item (Capítulo \ref{cap_fundamentacao-teorica} ), onde é mostrado o estado atual através de uma Revisão Sistemática da Literatura. Mostra ao leitor uma linha do tempo sobre a evolução do tema pesquisado nos últimos cinco anos.
    \item (Seção \ref{sec-arquitetura-transformer}), em que é documentada a arquitetura \textit{Transformer}, explorando seu mecanismo de atenção, codificador e decodificador, e outros componentes relevantes nesta arquitetura.
    \item (Seção \ref{cap-anal-pre-trans-ts}), onde é abordada a aplicação da arquitetura proposta com foco em \textit{forecast} para um conjunto de dados que contém métricas de um servidor real, armazenadas e coletadas em formato de série temporal.
    \item (Capítulo \ref{Cap_metodologia}), é documentado toda metodologia, detalhes do conjunto de dados, estratégia de pré-processamento e métricas utilizadas para avaliação da arquitetura.
    \item  (Capítulo \ref{Cap_resultados-preliminares}), neste capítulo são abordados os resultados parciais obtidos até o presente momento, contextualização e objetivos, e a configuração experimental.
    \item (Capítulo \ref{Cap_conclusao}), é apresentada a conclusão parcial, cronograma contendo as atividades a serem realizadas até a defesa desta dissertação, como a publicação de artigos relacionados ao tema, além da evolução do modelo e documentação dos resultados obtidos. 
\end{itemize}

A problemática que pautou o desenvolvimento desta Dissertação de Mestrado foi: uma arquitetura de \textit{deep learning} baseada em \textit{transformers}, aplicada à análise de métricas de dispositivos de ambientes de TI, pode contribuir de maneira proativa para a melhoria da eficiência operacional e resiliência?

Esta pesquisa é justificada pela crescente complexidade dos sistemas de TI e pelo aumento volumétrico de dados gerados, desafiando as práticas operacionais tradicionais. Atualmente, muitas operações de TI ainda são executadas manualmente e adotam uma abordagem reativa, o que pode ser inadequado diante da criticidade e da dinâmica dos sistemas modernos. A automação e a proatividade, facilitadas pelo uso de \textit{machine learning}, são essenciais para a gestão eficiente desses ambientes. A análise de séries temporais, em particular, oferece um meio robusto para a detecção precoce de anomalias e a previsão de tendências, contribuindo para a prevenção de falhas e a otimização de recursos. Portanto, esta dissertação contribui para o campo acadêmico ao explorar a aplicabilidade de técnicas avançadas de aprendizado de máquina em um contexto operacional crítico, fornecendo \textit{insights} valiosos para a evolução das práticas de AIOps.