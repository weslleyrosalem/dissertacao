\chapter{Conclusão}
\label{conclusoes}

A evolução das redes de computadores em virtude da popularização de acesso à Internet faz com que os dados trafegados pela rede mundial e pelas redes corporativas ou pessoais possuam uma complexidade no tocante a anatomia dos dados. Outro ponto relevante é que o crescimento do número de dispositivos que se conectam à web torna a quantidade de dados disponíveis para análise um fator determinante na escolha de métodos adequados para redução do custo computacional de processamento.

O processamento de tais dados pode ter diferentes objetivos. A área de \textit{Data Mining} lida com essa tarefa, empregando técnicas e ferramentas capazes de processar grandes quantidades de dados. A extração adequada de tais dados pode permitir que métodos matemáticos/estatísticos consigam reconhecer padrões e, com o emprego de algoritmos de \textit{Machine Learning}, classificar os dados de acordo com alguns objetivos hipotéticos. Neste sentido, a área de Segurança da Informação, em especial no que tange às Redes de Computadores, tem se beneficiado destes supracitados algoritmos, técnicas e ferramentas na construção de Sistemas de Detecção de Intrusão mais robustos.

A presente Dissertação de Mestrado justificou o emprego dos métodos apresentados por meio de uma análise do estado-da-arte obtido pela execução de uma Revisão Sistemática da Literatura que mostrou uma tendência da utilização de quatro métodos de classificação e um método específico de Ensemble (\textit{committees} por meio de \textit{Stacking}). Ademais, para escolha adequada de quais classificadores fariam parte de um \textit{Ensemble} que entregasse a melhor \textit{performance} possível, observando-se o fator custo X benefício (processamento X resultados), fora empregada uma técnica de nominada \textit{Diversity Pruning} que se mostrou ótima na detecção em cinco diferentes \textit{datasets} de seis analisados. Faz-se importante destacar que o método empregado não é indicado na detecção de ataques DDoS pois obteve desempenho inferior a métodos individuais.


Uma comparação dos resultados obtidos via \textit{Stacking} por meio de \textit{Diversity Pruning} com trabalhos correlatos que tenham utilizada a mesma técnica de \textit{Ensemble} demonstrou que a metodologia empregada nesta pesquisa foi superior em todos os cenários.

Conclui-se, portanto, que a escolha dos classificadores observando-se a técnica de \textit{Diversity Pruning} e a composição de um \textit{Committee} por meio de \textit{Stacking} é adequada para detecção de intrusão em redes de computadores. Os métodos empregados produziram resultados ótimos em cinco dos seis \textit{datasets} processados, sendo, portanto, este método adequado para detecção de ataques de \textit{portscan}, \textit{infiltration}, Web,  \textit{brute-force} e de ataques provenientes de \textit{Botnets} e não adequado para detectar ataques DDoS. 

Faz-se importante ressaltar que durante a execução das disciplinas obrigatórias, além da disciplina de Estudos Especiais e do desenvolvimento da Dissertação para a qualificação e defesa, o aluno autor deste documento também conseguiu algumas publicações listadas a seguir:

\begin{enumerate}
 
 
     \item ; LUCAS, T. J.; DA COSTA, K. A. P.; PAPA, J. P.; PIRES, R. G.; TOJEIRO, C. A. C. \textit{\textbf{Machine Learning for Web Intrusion Detection: A Comparative Analysis of Feature Selection Methods mRMR and PFI}}. In: \textit{Springer Artificial Intelligence and Soft Computing}. \textit{The 18th International Conference on Artificial Intelligence and Soft Computing}. [aprovada, mas aguardando evento que foi adiado devido ao COVID-19].
 
    \item DA COSTA, K. A. P.; DAS NEVES, M. J.; LUCAS, T. J.; RACY, C. H. A. \textbf{Estatísticas para Detecção de Bots em Redes Sociais}. In: \textit{International Association for Development of the Information Society}, IADIS. 2019. IV Conferência Ibero Americana de Computação Aplicada (CIACA). 2019. p. 174-190.
    

    
    \item LUCAS, T. J.; DA COSTA, K. A. P.; FERREIRA, G. V.; FERREIRA, M. R.; PLACHTA, R. D. \textit{\textbf{Non-fragmented Network Flow Design Analysis: Comparison IPv4 with IPv6 Using Path MTU Discovery}}. ACM \textit{International Journal of Network Management}. MDPI Computers, v. 9, p. 54, 2020.
    
    
    \item LUCAS, T. J.; DA COSTA, K. A. P.; DAS NEVES, M. J.; DE ANDRADE, V. R.; FERRARI, R. M. \textbf{Redirecionamento de Ataques de Negação de Serviço pós-Anti-Honeypot em Ambientes SDN.}. In: \textit{International Association for Development of the Information Society}, IADIS. 2020. VII Conferência Ibero Americana de Computação Aplicada (CIACA). 2020. [aprovada, aguardando conferência que irá ocorrer em novembro/2020].

    
\end{enumerate}

Como sugestão para futuras pesquisas fica a possibilidade de sem realizar testes com outros classificadores objetivando criar diferentes modelos de \textit{Stacking} utilizando \textit{Pruning} por diferentes técnicas. Além disso é relevante executar testes em outros \textit{datasets} para análise de desempenho.
