\chapter{Considerações Parciais} \label{cap:conclusion}

Através da observação das tendências extraídas de trabalhos correlatos e documentadas na Seção de Revisão Sistemática da Literatura (\ref{cap_revisao-sistematica}), aplicação de técnicas de \textit{machine learning} são cada vez mais utilizadas no contexto de operações de T.I, buscando não apenas automatizar tarefas, mas também identificar causa raiz de problemas complexos, prever tendência de consumo de recursos e identificar anomalias.

A utilização de uma arquitetura Transformer para análise preditiva em séries temporais com métricas de consumo dos dispositivos, mostra-se promissora, dado sua capacidade de capturar dependências de longo alcance e tendências complexas, diferenciando anomalias de sazonalidades, tornando-se em uma ferramenta muito relevante para AIOps, ajudando a melhorar a eficiência e a confiabilidade dos sistemas.


\section{Cronograma}
Apresenta-se a Tabela \ref{tab:cronograma} como cronograma até a defesa:

\begin{table}[h]
\begin{tabular}{|c|c|c|c|c|c|c|c|c|c|c|c|}
\hline
 & Ago & Set & Out & Nov & Dez & Jan & Fev & Mar & Abril & Maio & Jun \\ \hline
\begin{tabular}[c]{@{}c@{}}Correções \\ Pontuadas\\ pela Banca\end{tabular} & X & X &  &  &  &  &  &  &  &  &  \\ \hline
\begin{tabular}[c]{@{}c@{}}Metodologia / \\ Experimentos\end{tabular} & X & X & X &  &  &  &  &  &  &  &  \\ \hline
Desenvolvimento & X & X & X & X &  &  &  &  &  &  &  \\ \hline
Qualificação &  &  &  &  & X &  &  &  &  &  &  \\ \hline
\begin{tabular}[c]{@{}c@{}}Escrita e Submissão\\ de Artigos\end{tabular} &  &  &  &  &  & X & X & X & X & X &  \\ \hline
Defesa &  &  &  &  &  &  &  &  &  &  & X \\ \hline
\end{tabular}
\caption{Cronograma de Trabalho até a Defesa}
\label{tab:cronograma}
\end{table}