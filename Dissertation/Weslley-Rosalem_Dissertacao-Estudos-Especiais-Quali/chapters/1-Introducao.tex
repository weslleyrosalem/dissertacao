\chapter{INTRODUÇÃO}

Na era da transformação digital, os ambientes de Tecnologia da Informação (TI) tornaram-se incrivelmente complexos e dinâmicos. As empresas estão cada vez mais dependentes de infraestruturas de TI para executar e gerenciar suas operações diárias \cite{pahl2015containerization}. Esta dependência está levando as organizações a implementar sistemas de grande escala que são difíceis de monitorar e manter. 

O gerenciamento desses grandes ambientes de TI é um desafio significativo devido à sua natureza heterogênea e à variedade de componentes de hardware e software que eles incorporam. Os ambientes de TI modernos são constituídos por diversos sistemas, incluindo servidores físicos, máquinas virtuais, contêineres, redes, bancos de dados, e aplicações \cite{pahl2015containerization}. Além disso, com a adoção crescente de modelos de computação em nuvem, os recursos de TI estão cada vez mais distribuídos geograficamente.

Dentro deste contexto, as operações de TI enfrentam o desafio de garantir a alta disponibilidade e desempenho desses sistemas. Uma das tarefas mais difíceis em gerenciamento de TI, é a identificação de causa raiz, que envolve determinar a origem de problemas ou falhas em um sistema. A identificação de causa raiz é especialmente difícil devido à interdependência dos componentes e à enorme quantidade de dados gerados pelos sistemas \cite{cohen2004correlating}.

A manutenção também é uma área que apresenta desafios únicos. Manter os sistemas atualizados e garantir que eles estejam seguros contra vulnerabilidades exige um monitoramento contínuo e muitas vezes envolve a coordenação de mudanças em vários sistemas. Além disso, o processo de manutenção deve ser realizado de forma a minimizar o impacto sobre os serviços críticos de negócio.

Para lidar com esses desafios, as empresas estão procurando soluções que possam automatizar e otimizar suas operações. A Inteligência Artificial para Operações de TI (AIOps), surgiu como um campo que aplica técnicas de Inteligência Artificial (IA) para resolver estes problemas de Operações (OPs). AIOps combina \textit{big data}, e, aprendizado de máquina, para automatizar o gerenciamento de TI, melhorando a detecção de anomalias, previsão de capacidade, correlação de eventos, e identificação de causa raiz \cite{sill2019aiops}.

Dentro de \textit{AIOps}, uma abordagem promissora é o uso de arquiteturas de \textit{Transformer} para análise de séries temporais, como métricas de desempenho de sistemas. Essa arquitetura, originalmente desenvolvidas para tarefas de processamento de linguagem natural, têm demonstrado ser eficazes em capturar dependências de longo alcance e modelar complexidades em séries temporais \cite{vaswani2017attention}. 

O \textit{Prometheus}, um sistema de monitoramento e banco de dados de séries temporais, é frequentemente usado para coletar métricas de sistemas de TI. Integrando uma arquitetura de \textit{Transformer} com métricas coletadas pelo \textit{Prometheus}, pode-se desenvolver modelos de aprendizado de máquina mais precisos e eficientes para análise preditiva e detecção de anomalias.

Nesta dissertação, será explorarado como uma arquitetura de \textit{Transformer} pode ser aplicada a conjuntos de dados de séries temporais coletados com o \textit{Prometheus} para ajudar a identificar anomalias e prever tendências de consumo de recursos em ambientes de TI. Esta abordagem tem o potencial de melhorar significativamente a eficácia das operações de TI, tornando mais fácil gerenciar a complexidade crescente dos ambientes modernos. Mais adiante, iremos abordar os fundamentos teóricos por trás das tecnologias envolvidas, descrever a metodologia utilizada e discutir os resultados obtidos.




















