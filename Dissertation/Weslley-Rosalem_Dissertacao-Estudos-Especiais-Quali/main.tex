%
% A Universidade Estadual Paulista "Júlio de Mesquita Filho"            
% MODELO DE  RELATÓRIO 
%
\documentclass[
	% -- opções da classe memoir --
	12pt,				% tamanho da fonte
	openright,			% capítulos começam em pág ímpar (insere página vazia caso preciso)
	oneside,			% para impressão em verso e anverso. Oposto a oneside
	a4paper,			% tamanho do papel. 
	% -- opções da classe abntex2 --
	%chapter=TITLE,		% títulos de capítulos convertidos em letras maiúsculas
	%section=TITLE,		% títulos de seções convertidos em letras maiúsculas
	%subsection=TITLE,	% títulos de subseções convertidos em letras maiúsculas
	%subsubsection=TITLE,% títulos de subsubseções convertidos em letras maiúsculas
	% -- opções do pacote babel --
	english,			% idioma adicional para hifenização
	french,				% idioma adicional para hifenização
	spanish,			% idioma adicional para hifenização
	brazil				% o último idioma é o principal do documento
	]{abntex2}


% Pacotes básicos 
\usepackage[alf]{abntex2cite}
\usepackage{chngcntr}
\counterwithout{footnote}{chapter}
\counterwithout{equation}{chapter}
\usepackage{lmodern}			% Usa a fonte Latin Modern			
\usepackage[T1]{fontenc}		% Selecao de codigos de fonte.
\usepackage[utf8]{inputenc}		% Codificacao do documento (conversão automática dos acentos)
\usepackage{lastpage}			% Usado pela Ficha catalográfica
\usepackage{indentfirst}		% Indenta o primeiro parágrafo de cada seção.
\usepackage{color}				% Controle das cores
\usepackage{graphicx}			% Inclusão de gráficos
\usepackage{microtype} 			% para melhorias de justificação
\usepackage{afterpage}
\usepackage{amsmath}
\usepackage{amssymb,url}
\usepackage{hhline}
\usepackage{xcolor,tikz,bm,colortbl}
\usepackage[ruled,linesnumbered,vlined,portuguese,onelanguage]{algorithm2e}

%Weslley-lars
\usepackage{float}
\usepackage{longtable,ltcaption} % para as tabelas
\usepackage{hyperref} %citacoes customizadas
\usepackage{rotating}
\usepackage{pdflscape}
\usepackage{longtable}
\usepackage{adjustbox}
\usepackage{adjustbox}
\usepackage{lipsum}
\usepackage{pdfpages}
\usepackage{longtable}
%\usepackage{biblatex}
%\usepackage[style=numeric]{biblatex}  % Include the biblatex package



% Algumas mudanças devem ser realizadas neste arquivo, por exemplo, financiadora do projeto
\usepackage{0-libs/customizacoes}


% Informações de dados para CAPA e FOLHA DE ROSTO
\titulo{AIOPs - Utilizando uma Arquitetura Transformer em Time Series com Prometheus}
\autor{Weslley Rosalem}
\local{Bauru}
\data{2023}
\orientador{Prof. Dr. Kelton Augusto C. Pontara}
\instituicao{}
\tipotrabalho{Dissertação de Mestrado Acadêmico}
% O preambulo deve conter o tipo do trabalho, o objetivo, 
% o nome da instituição e a área de concentração 
\preambulo{Trabalho apresentado para disciplina de estudos especiais, parte do  Programa de Pós-Graduação em Ciência da Computação, do Instituto de Biociências, Letras e Ciências Exatas da Universidade Estadual Paulista ``Júlio de Mesquita Filho", Câmpus de Bauru.}
% ---

% Configurações de aparência do PDF final

% alterando o aspecto da cor preta
\definecolor{black}{RGB}{0,0,0}
\definecolor{blue}{RGB}{0,0,190}

% informações do PDF
\makeatletter
\hypersetup{
     	%pagebackref=true,
		pdftitle={\@title}, 
		pdfauthor={\@author},
    	pdfsubject={\imprimirpreambulo},
	    pdfcreator={LaTeX com abnTeX2},
		pdfkeywords={abnt}{latex}{abntex}{abntex2}{dissertação}, 
		colorlinks=true,       		% false: boxed links; true: colored links
    	linkcolor=blue,          	% color of internal links
    	citecolor=blue,        		% color of links to bibliography
    	filecolor=magenta,      		% color of file links
		urlcolor=black,
		bookmarksdepth=4
}
\makeatother

% O tamanho do parágrafo é dado por:
\setlength{\parindent}{1.3cm}

% Controle do espaçamento entre um parágrafo e outro:
\setlength{\parskip}{0.2cm}  % tente também \onelineskip

% compila o indice
\makeindex
% ---

% ----
% Início do documento
% ----
\begin{document}

% Seleciona o idioma do documento (conforme pacotes do babel)
%\selectlanguage{english}
%\selectlanguage{brazil}


% Retira espaço extra obsoleto entre as frases.
\frenchspacing % ----------------------------------------------------------
% ELEMENTOS PRÉ-TEXTUAIS
% ----------------------------------------------------------
% Capa
\imprimircapa
% ---
% Folha de rosto
% (o * indica que haverá a ficha bibliográfica)
\imprimirfolhaderosto*


% inserir lista de abreviaturas e siglas
% ---
\begin{siglas}
\item[AIOPs] - \textit{Artificial Intelligence for IT Operations}
\item[ANN] - \textit{Artificial Neural Network}
\item[CPU] - \textit{Central Processing Unit}
\item[DBSCAN] - \textit{Density-based spatial clustering of applications with noise}
\item[DP] - \textit{Deep Learning}
\item[IDS] - \textit{Intrusion Detection System}
\item[LSTM] - \textit{Long short-term memory}
\item[ML] - \textit{Machine Learning}
\item[MLOPs] - \textit{Machine Learning Operations}
\item[NN] - \textit{Neural Network}
\item[NNET] - \textit{Neural Network}
\item[SMOTE] - \textit{Synthetic Minority Over-sampling Technique}
\item[RNN] - \textit{Recurrent neural network}
\item[SVM] - \textit{Support Vector Machine}
\item[TF] - \textit{Transformer}
\item[TS] - \textit{Time Series}
\item[TSTF] - \textit{Time Series Transformer}
\end{siglas}
% ---

% inserir o sumario
\pdfbookmark[0]{\contentsname}{toc}
\tableofcontents
\cleardoublepage

% ----------------------------------------------------------
% ELEMENTOS TEXTUAIS
% ----------------------------------------------------------
\pagestyle{simple}

% ----------------------------------------------------------
%\input{chapters/1-resumo.tex}
%\input{chapters/2-ex1}
%\input{chapters/3-ex2}
%\input{chapters/4-ex3}
%\input{chapters/5-conclusao}
\chapter{INTRODUÇÃO} \label{cap_introducao}

Em meio à era da transformação digital, os ambientes de Tecnologia da Informação (TI), evoluíram para atender diversos cenários críticos, de diferentes setores, incluindo mas não limitando-se a: saúde, financeiro, telecomunicações, segurança, logística, entre outros. Visando melhorar a disponibilidade, resiliência,  flexibilidade e confiabilidade, eles também tornaram-se notavelmente mais complexos e dinâmicos. A crescente dependência das empresas em infraestruturas de TI para conduzir suas operações diárias, tem impulsionado a implementação de sistemas distribuídos em larga escala que são intrinsecamente desafiadores de monitorar, escalar, manter e prever a quantidade de recursos necessários \cite{pahl2015containerization}.

O gerenciamento desses vastos ambientes de TI é desafiador, dada a sua natureza heterogênea e a multiplicidade de componentes de hardware e software que integram. Os ambientes de TI contemporâneos englobam uma variedade de sistemas, desde servidores físicos, máquinas virtuais, contêineres, redes, bancos de dados e aplicações \cite{pahl2015containerization}. A adoção crescente da computação em nuvem amplia ainda mais essa complexidade, com recursos de TI distribuídos geograficamente.

Nesse cenário, as operações de TI são pressionadas a garantir a disponibilidade e o ótimo desempenho  desses sistemas. Uma das tarefas mais árduas é a identificação de causa raiz, que busca determinar a origem de falhas ou problemas em um sistema. Esta tarefa é exacerbada pela interdependência dos componentes e pela vastidão de dados gerados \cite{cohen2004correlating}.

A manutenção por sua vez, apresenta seus próprios desafios. Assegurar sistemas atualizados e protegidos contra vulnerabilidades, demanda monitoramento contínuo e frequentemente, a coordenação de atualizações em múltiplos sistemas. O processo de manutenção precisa ser meticuloso para minimizar interrupções em serviços  críticos.

Diante desses desafios, as organizações buscam soluções que possam automatizar e otimizar suas operações. A Inteligência Artificial para Operações de TI, conhecida como \textit{Artificial Inteligence for I.T Operations} (AIOps), emerge como uma disciplina que emprega técnicas de IA para abordar problemas operacionais. AIOps integra \textit{big data}\footnote{\textit{Big Data} refere-se a conjuntos de dados que são tão grandes e complexos que exigem métodos avançados e ferramentas de análise para processar. Estes dados podem ser estruturados ou não estruturados e são caracterizados por volume, velocidade e variedade.}
 e aprendizado de máquina, visando aprimorar aspectos como detecção de anomalias, previsão de capacidade, correlação de eventos e identificação de causas raízes \cite{sill2019aiops}.

No âmbito do AIOps, destaca-se o uso de aprendizado profundo\footnote{Aprendizado profundo, ou \textit{deep learning}, é uma subárea da aprendizagem de máquina \textit{(machine learning)}, que por sua vez é um domínio da inteligência artificial (IA). Essa técnica se inspira na estrutura e funcionamento do cérebro humano, mais especificamente nas redes neurais biológicas, para processar dados e criar padrões para a tomada de decisões.} para análise de séries temporais\footnote{Uma série temporal é uma sequência de pontos de dados coletados ou registrados em intervalos de tempo regulares ou irregulares. Cada ponto de dados na série temporal é um registro de alguma atividade, observação ou medição em um momento específico. Tecnicamente, uma série temporal é uma sequência de observações  x(t), onde  (t) representa o tempo em que a observação foi feita.}, como métricas de desempenho (KPI). A arquitetura \textit{Transformer}, originalmente concebida para processamento de linguagem natural, tem se mostrado eficaz em modelar complexidades em séries temporais, capturando dependências de longo alcance \cite{vaswani2017attention}.

Ademais, o \textit{Prometheus}, um sistema de monitoramento e banco de dados de séries temporais, é amplamente adotado para coletar métricas de sistemas de TI. Ao integrar arquitetura \textit{Transformer} com métricas do \textit{Prometheus}, é possível desenvolver modelos de aprendizado de máquina mais robustos para análise preditiva e detecção de anomalias.

Neste trabalho, será explorado a aplicação de \textit{Transformer} em dados de séries temporais obtidos via \textit{Prometheus}, visando identificar anomalias e prever tendências de consumo de recursos. Esta abordagem tem o potencial de aprimorar substancialmente a eficiência das operações de TI, facilitando o gerenciamento da crescente complexidade dos ambientes atuais. Nos capítulos subsequentes, serão abordados os fundamentos teóricos das tecnologias em foco, a metodologia adotada e os resultados alcançados, bem como a estrutura organizacional da pesquisa, sendo:

\begin{itemize}
    \item (Capítulo \ref{cap_introducao}), em que é apresentado o tema objeto da pesquisa de forma com que sejam justificados os esforços na realização da presente dissertação, bem como realizada a contextualização, apresentados os objetivos geral e específicos e a metodologia empregada;
    \item (Capítulo \ref{cap_fundamentacao-teorica} ), onde é mostrado o estado atual através de uma Revisão Sistemática da Literatura. Mostra ao leitor uma linha do tempo sobre a evolução do tema pesquisado nos últimos cinco anos.
    \item (Seção \ref{sec-arquitetura-transformer}), em que é documentada a arquitetura \textit{Transformer}, explorando seu mecanismo de atenção, codificador e decodificador, e outros componentes relevantes nesta arquitetura.
    \item (Seção \ref{cap-anal-pre-trans-ts}), onde é abordada a aplicação da arquitetura proposta com foco em \textit{forecast} para um conjunto de dados que contém métricas de um servidor real, armazenadas e coletadas em formato de série temporal.
    \item (Capítulo \ref{Cap_metodologia}), é documentado toda metodologia, detalhes do conjunto de dados, estratégia de pré-processamento e métricas utilizadas para avaliação da arquitetura.
    \item  (Capítulo \ref{Cap_resultados-preliminares}), neste capítulo são abordados os resultados parciais obtidos até o presente momento, contextualização e objetivos, e a configuração experimental.
    \item (Capítulo \ref{Cap_conclusao}), é apresentada a conclusão parcial, cronograma contendo as atividades a serem realizadas até a defesa desta dissertação, como a publicação de artigos relacionados ao tema, além da evolução do modelo e documentação dos resultados obtidos. 
\end{itemize}

A problemática que pautou o desenvolvimento desta Dissertação de Mestrado foi: uma arquitetura de \textit{deep learning} baseada em \textit{transformers}, aplicada à análise de métricas de dispositivos de ambientes de TI, pode contribuir de maneira proativa para a melhoria da eficiência operacional e resiliência?

Esta pesquisa é justificada pela crescente complexidade dos sistemas de TI e pelo aumento volumétrico de dados gerados, desafiando as práticas operacionais tradicionais. Atualmente, muitas operações de TI ainda são executadas manualmente e adotam uma abordagem reativa, o que pode ser inadequado diante da criticidade e da dinâmica dos sistemas modernos. A automação e a proatividade, facilitadas pelo uso de \textit{machine learning}, são essenciais para a gestão eficiente desses ambientes. A análise de séries temporais, em particular, oferece um meio robusto para a detecção precoce de anomalias e a previsão de tendências, contribuindo para a prevenção de falhas e a otimização de recursos. Portanto, esta dissertação contribui para o campo acadêmico ao explorar a aplicabilidade de técnicas avançadas de aprendizado de máquina em um contexto operacional crítico, fornecendo \textit{insights} valiosos para a evolução das práticas de AIOps.
\chapter{Revisão Sistemática da Literatura}
\label{cap_revisao-sistematica}

A Revisão Sistemática da Literatura, é uma metodologia de pesquisa que identifica, avalia e interpreta todas as pesquisas relevantes disponíveis sobre um tópico particular. Ela é caracterizada por sua abordagem rigorosa e claramente definida, que pode ser replicada e auditada, assegurando assim uma maior confiabilidade ao processo \cite{tranfield2003systematic}.

Esta metodologia é usada principalmente para coletar e sintetizar evidências empíricas que se ajustem a critérios de inclusão pré-definidos \cite{kitchenham2007guidelines}. Ela envolve a identificação de questões de pesquisa relevantes, a seleção e avaliação qualitativa de estudos, a extração de dados e, a síntese e apresentação dos resultados.


\section{Técnicas para Revisão Sistemática da Literatura}
\label{subcap_tec_rev-sistematica}

Este é um método indispensável na área de pesquisa acadêmica, visto que permite uma visão ampla e abrangente do conhecimento atual, identificando \textit{gaps} que podem ser explorados em futuros trabalhos \cite{petticrew2006systematic}. Além disso, ela reduz a duplicação de esforços, dando visibilidade às pesquisas já realizadas sobre o tema.

Em sua elaboração, deve-se começar com a definição de um protocolo de pesquisa, onde são determinados os critérios de inclusão e exclusão, as bases de dados a serem pesquisadas, e as estratégias de busca a serem utilizadas \cite{biolchini2005systematic}.  A Figura \ref{fig:processo_rev_sistematica} ilustra o procedimento utilizado neste trabalho:


\begin{figure}[H]
\centering
\caption{Procedimento adotado para realização da Revisão Sistemática da Literatura.} \includegraphics[width=8cm,height=\textwidth,keepaspectratio]{2-images/Fluxograma-artigos.jpg}
\newline \centering{Fonte: Elaborado pelo autor}\label{fig:processo_rev_sistematica}
\end{figure}


Detalham-se as etapas ilustradas na Figura \ref{fig:processo_rev_sistematica} conforme segue:

\begin{enumerate}
    
    \item As \textit{queries} são \textit{strings} que definem a lógica para seleção dos artigos. Para a presente pesquisa procurou-se buscar artigos cujo título ou \textit{keywords} contivessem as palavras \textit{AIOPs} além de limitar o intervalo de idade da publicação entre o ano de 2018 ao ano de 2023. Importante destacar que para o ano de 2023 foram considerados os seis primeiros meses.
    
    \item Por conhecida relevância na área da Ciência da Computação e, em especial, na área de Redes de Computadores, monitoramento e observabilidade, duas bases de pesquisa científica foram consideradas para execução das \textit{queries} em seus respectivos motores de busca: IEEExplore\footnote{https://ieeexplore.ieee.org/} e ACM \textit{Digital Library}\footnote{https://dl.acm.org/}.
    
    \item Após execução das \textit{queries} nas bases de pesquisa supracitadas foram obtidos 80 artigos. Uma distribuição da quantidade de artigos obtidos por ano de publicação pode ser observada na Figura \ref{fig:grafico_num_papers}:
      \begin{figure}[H]
    \centering
    \caption{Distribuição dos artigos obtidos pelo ano de publicação.} \includegraphics[width=12cm,height=\textwidth,keepaspectratio]{2-images/artigos-bars-horizontal.png}
    \newline \centering{ Fonte: Elaborado pelo autor}\label{fig:grafico_num_papers}
    \end{figure}
    \item A etapa de filtragem consistiu em eliminar artigos duplicados ou inacessíveis (22 em duplicidade e 2 sem acesso), e, posteriormente, realizou-se a leitura dos 56 artigos restantes onde observou-se que 35 não tinham relação direta com a linha de pesquisa objetivada nesta Dissertação de Mestrado. Portanto, após a aplicação dos filtros, 21 artigos foram identificados como fortemente relacionados ao tema objeto. A Figura \ref{fig:grafico_processamento_papers} resume a corrente etapa:
    
        \begin{figure}[H]
    \centering
    \caption{Detalhamento do resultado após os processos de filtragem dos artigos.} \includegraphics[width=10cm,height=\textwidth,keepaspectratio]{2-images/chart-2.png}
    \newline \centering{ Fonte: Elaborado pelo autor}\label{fig:grafico_processamento_papers}
    \end{figure}    

    \item Por fim, a quinta etapa consistiu na extração dos dados mais relevantes dos artigos selecionados. Foram extraídos para fins de observação de tendências as seguintes informações:
     \begin{itemize}
         \item Tipo de modelo: quais as técnicas de machine learning foram utilizadas ou sugeridas;
         \item \textit{Dataset}: quais as bases de dados usadas para realização dos testes;
         \item Arquitetura: quais arquiteturas proposta;
         \item \textit{Feature selection}: quais  \textit{features} foram utilizadas;
         \item Resultados: a quais resultados os autores chegaram, como RMSE e MAE; e
         \item Metodologia: quais materiais foram usados no processo de criação do modelo de  identificação de anomalias e predição (linguagens e softwares).
     \end{itemize}

\end{enumerate}

Os artigos selecionados são resumidos de forma a destacar os elementos mais importantes para relacionamento com a presente pesquisa. Os estudos foram organizados seguindo o seguinte critério:

\begin{itemize}
    \item Detalhamento dos trabalhos correlatos para o ano de 2018 na Subseção \ref{trab_correlatos_18};
    
    \item Detalhamento dos trabalhos correlatos para o ano de 2019 na Subseção \ref{trab_correlatos_19};
    
    \item Detalhamento dos trabalhos correlatos para o ano de 2020 na Subseção \ref{trab_correlatos_20};
    
    \item Detalhamento dos trabalhos correlatos para o ano de 2021 na Subseção \ref{trab_correlatos_21};
    
    \item Detalhamento dos trabalhos correlatos para o ano de 2022 na Subseção \ref{trab_correlatos_22};
    
    \item Detalhamento dos trabalhos correlatos para o ano de 2023 na Subseção \ref{trab_correlatos_23};
    
    
\end{itemize}

Para ilustrar todo o processo de Revisão Sistemática da Literatura, desde os detalhes das \textit{queries} até a extração dos dados mais relevantes, a Figura \ref{fig:diagrama_detalhado_rev_sistematica} exibe um fluxograma completo do processo utilizado:

\begin{figure}[H]
\centering
\caption{Fluxo detalhado do procedimento realizado na Revisão Sistemática da Literatura. } \includegraphics[width=\textwidth,keepaspectratio]{2-images/Fluxograma-papers.png}
\newline \centering{Fonte: Elaborado pelo autor}\label{fig:diagrama_detalhado_rev_sistematica}
\end{figure}    
    
    

\subsection{Trabalhos Correlatos - ano de 2018}\label{trab_correlatos_18}
Não foram encontrados trabalhos neste período.

\subsection{Trabalhos Correlatos - ano de 2019}\label{trab_correlatos_19}

Os autores \cite{8752866}, discutem no artigo (\textit{Anomaly Detection and Classification using Distributed Tracing and Deep Learning}) , a aplicação de Inteligência Artificial (IA) nas Operações de TI \textit{(AIOps)} para detectar anomalias com base em registros de \textit{distributed trancing}, que contêm informações detalhadas sobre a disponibilidade e o tempo de resposta dos serviços. Os autores propõem a detecção de anomalias no tempo de resposta, com aprendizado não supervisionado, baseada em técnicas de modelagem de dados de aprendizado profundo, e avalia a precisão e o desempenho da abordagem tanto em um ambiente de teste experimental quanto em uma nuvem em ambiente de produção em grande escala. Eles também mencionam as vantagens de combinar \textit{GRUs} (Unidades Recorrentes de Memória) e \textit{autoencoders} variacionais para aprender múltiplas distribuições complexas de dados em séries temporais geradas por sistemas de rastreamento distribuído.


\subsection{Trabalhos Correlatos - ano de 2020}\label{trab_correlatos_20}


Os autores \cite{9156101} ,  abordam o problema de monitoramento de um ambiente de TI complexo, incluindo nuvens privadas e públicas, ambientes de IoT, aplicativos e contêineres. A necessidade de um modelo de contexto nas Operações de TI é demonstrada para gerenciar as enormes quantidades de dados, armazenados em vários tipos de formatos e fornecer uma visão holística do ambiente de TI. A arquitetura proposta por eles trata-se de um \textit{framework} composto por cinco camadas: aquisição, gerenciamento, análise, apresentação de dados e respostas automatizadas. O \textit{Monitoring Resource Model} \textit{(MRM)} está no centro desse \textit{framework} e suporta a aquisição, o gerenciamento e a análise de dados de contexto, bem como uma camada de ação para fins de alerta e automação. Para isso, é proposto um modelo de aprendizado de máquina de redes neurais, o\textit{ Long-Short-Term-Memory (LSTM)}. Esses modelos são empregados para aprendizado supervisionado, a fim de prever o comportamento futuro dos sistemas monitorados.


\cite{9311349} Explicam sobre como a correlação, aprendizado de máquina e análise de big data podem reduzir o número de incidentes em infraestruturas de TI convergentes. O estudo enfatiza a importância do uso de técnicas de aprendizado de máquina, especialmente no contexto de (AIOps), para descobrir relacionamentos entre objetos e processos, e identificar padrões e sequências em milhões de eventos. A arquitetura proposta envolve a utilização de análise e aprendizado de máquina para investigar grandes quantidades de dados acumulados por diversos instrumentos e dispositivos de TI. Eles também mencionam a necessidade de dados de treinamento inicial representativos e pesquisas adicionais para melhorar as operações de TI em infraestruturas convergentes.

O objetivo do estudo apresendado por \cite{9443765}, é propor uma rede neural dinâmica, para prever fluxos de dados de séries temporais em cenários de AIOps, adim de realizar a previsão de tendência em tempo real de fluxos de dados. Os modelos de aprendizado de máquina utilizados incluem \textit{MWNN (Multi-Way Neural Network), WNN (Wavelet Neural Network) e LSTM (Long Short-Term Memory) }. Os dados utilizados durante os testes incluem conjuntos de dados de CPUs com diferentes capacidade. Os resultados obtidos mostram uma comparação com o consumo de recursos para MWNN, WNN e LSTM quando alcançam o mesmo desempenho, medido pela métrica RMSE. Também são fornecidos valores específicos de MSE, RMSE e MAE para os algoritmos nos mesmos dataset de CPUs .



\subsection{Trabalhos Correlatos - ano de 2021}\label{trab_correlatos_21}

Em seu estudo, os autores \cite{9516546} propõem a combinação de métodos integrados múltiplos para prever a capacidade de recursos. O modelo de aprendizado de máquina base utilizado é LSTM (Long Short-Term Memory). Usando o resultado desse modelo de previsão integrado com base nos dados históricos,  o estudou mostrou resultados interessantes no \textit{forecast} de recursos de TI, como o uso de CPU, disco rígido, memória. Isso pode ajudar a otimizar a capacidade de gerenciar os recursos  para manter o sistema estável e confiável. 


\cite{9605403} discutem o conceito de Machine Reasoning (MR) e seu papel na melhoria das Operações de Inteligência Artificial (AIOps) para Redes Baseadas em Intenções. MR é um ramo da Inteligência Artificial (IA) que se baseia na captura do conhecimento humano usando linguagens semânticas para mecanizar e executar fluxos de trabalho com um Motor de Raciocínio de Máquina (MRE). Eles mostram como esta abordagem complementa a Aprendizagem de Máquina (ML) fornecendo  precisão para inferências baseadas no conhecimento capturado e é adequado para resolver problemas que exigem profundo conhecimento de domínio. O cenário de uso porposto pelos autores, incluem MR em AIOps para automatizar e melhorar o processo de identificação da causa raiz de problemas em rede de computadores.

%27
\cite{9678534} Propõem um método chamado AID (Aggregated Intensity of Dependency), para prever de forma eficiente a intensidade de dependências em sistemas em nuvem em larga escala. O cenário utilizado pelos autores, envolve medir a intensidade de dependências em um sistema em nuvem de produção utilizando modelo de aprendizado de máquina. Durante os testes apresentado, dois conjuntos de dados foram utilizados: um conjunto de dados simulados gerados a partir de um sistema de microsserviços de referência e um conjunto de dados industriais coletados de um sistema em nuvem de produção. Os resultados da avaliação experimental mostraram que o método proposto mede com precisão a intensidade das dependências e supera várias bases de comparação.


%31
No artigo abordado pelos autores \cite{9680514}, é mostrado o design e a implementação de recursos para migrar os dados de sistemas de monitoramento antigos de uma aplicação nativa em nuvem para instâncias do Prometheus com \textit{framework Ananke}, e permitir a integração das séries temporais armazenadas no Prometheus com ferramentas padrão disponíveis em bibliotecas Python para análise de dados . O estudo também propõe uma estratégia de dimensionamento automático com base na previsão de picos de tráfego para uma aplicação web usando o modelo Facebook Prophet. O cenário de uso envolveu monitorar e modelar aplicações nativas de nuvem, com foco em métricas de desempenho de aplicativos reaia e estratégias de otimização envolvendo big data e AI/ML em  feedback contínuo . Em relação aos dados utilizados nos testes, os autores fizeram uso das séries temporais do Prometheus com métricas de servidores web como \textit{wordpress} coletados pelos mesmos.


\subsection{Trabalhos Correlatos - ano de 2022}\label{trab_correlatos_22}

%34
\cite{9492267} apresentam o TSAGen, uma ferramenta de geração de séries temporais, que pode ser usada para gerar dados sintéticos, fornecendo uma fonte de dados confiável para os demais pesquisadores. O TSAGen permite que os usuários avaliem de forma abrangente o desempenho de algoritmos de detecção de anomalias com dados sintéticos. A ferramenta aborda desafios como gerar anomalias diversas, ajustar a gravidade das anomalias e controlar as características dos KPIs gerados. A arquitetura proposta do TSAGen segue um design modular aditivo, em que cada componente é gerado independentemente para permitir simples integração de módulos individuais adicionais, de acordo com o cenário de cada pesquisa ou algoritmo.

%36
No estudo proposto pelos autores \cite{9746242}, é proposto uma rede neural profunda, chamada CDX-Net, para previsão de séries temporais multivariadas, no campo de Inteligência Artificial para Operações de TI (AIOps). O caso de uso envolve a modelagem e análise de dados de monitoramento, que são frequentemente representados como Séries Temporais Multivariadas (MTS). A arquitetura proposta do CDX-Net inclui módulos como ASPP, SRM, CAM, GRU, transformador e AB, que visam melhorar os procedimentos de extração e fusão de características. Os pesquisadores utilizam em seus experimentos, um conjunto de dados de uma aplicação web e compara o desempenho do modelo proposto com outros métodos de referência para Time Series.


%37
\cite{9746416} apresentam uma solução para o Desafio ICASSP-SPGC-2022 AIOps, uma competição organizada pela universidaade de \textit{Honk Kong}, mais especificamente para inferência precisa de combinações em \textit{root cause analysis}. O documento descreve os desafios nos dados da competição e propõe uma estrutura para resolver o problema. A arquitetura proposta inclui um método para inferir se uma amostra inclui múltiplas causas-raiz e a introdução de \textit{TextCNN}. O artigo também menciona que o método obteve uma pontuação de 0,93 e ficou em 4º lugar no \textit{leaderboard}.


%38
O estudo dos autores \cite{9789784}, propõe um método para gerar relatórios de falhas legíveis por humanos em sistemas de TIC, aproveitando dados de séries temporais e metadados textuais de métricas, a fim de detectar falhas em um sistema-alvo. É utilizado modelo LSTM (Long Short-Term Memory) para séries temporais multivariadas, que serve como o método de referência para comparação. A arquitetura proposta utiliza dados de séries temporais e metadados como entrada e gera um relatório de falha em formato de texto. Durante os testes, os autores utilizaram conjuntos de dados coletados de um sistema de microsserviços implantado em um cluster Kubernetes, os experimentos avaliaram a eficácia e o desempenho do método proposto na detecção de falhas em comparação com o método de referência. Os resultados experimentais mostraram que o método proposto alcançou alta precisão na detecção de cenários específicos, como consumo excessivo de CPU, vazamento de memória, e dados de rede.


%40
\cite{9825776} propõem uma representação gráfica para o DevOps em aplicações baseadas em aprendizado de máquina, também conhecida como MLOps, eles discutem a fase de planejamento do MLOps, onde o problema a ser resolvido e os dados disponíveis são identificados, abordagens de análise de dados adequadas e algoritmos são selecionados. Também mencionam a fase de codificação, onde o sistema e o código de ML são implementados e validados, e a fase de validação, que envolve a avaliação do desempenho do modelo de ML com novos dados. O artigo destaca a necessidade de um processo de MLOps integrado e menciona desafios na adoção do MLOps na prática.

%41
\cite{9835411} sugerem um \textit{framework} de dimensionamento automático proativo chamado RobustScaler para o cenário de computação em nuvem. O objetivo deste estudo é desenvolver um framework de dimensionamento automático que possa gerar decisões de mudança e elasticidade, otimizando o equilíbrio entre custo e Qualidade de Serviço (QoS), e que seja robusto a ruídos, dados ausentes e anomalias.  O modelo proposto captura tanto a periodicidade quanto a estocasticidade das chegadas de consultas e é treinado usando um algoritmo de Método dos Multiplicadores de Direção Alternada (ADMM). A arquitetura proposta do framework consiste em quatro componentes principais: detecção de periodicidade, modelagem histórica de chegadas de consultas, previsão de chegada de consultas e plano de dimensionamento 

%43
Neste artigo, os autores \cite{9892264}, apresentam um framework baseado em aprendizado federado chamado EFL-WP para previsão de carga de trabalho em ambientes de inter-nuvem. O framework proposto tem como objetivo colaborar para treinar modelos de machine learning para previsão de carga de trabalho sem compartilhar informações sensiveis. Para isso os autores sugerem p uso de modelos LSTM para prever a utilização da CPU, utilização de memória e outras métricas de desempenho. A arquitetura proposta consiste em um coordenador e treinadores, onde o coordenador agenda tarefas de treinamento, orquestra os treinadores e mantém modelos globais, enquanto os treinadores usam seus próprios dados para treinar modelos locais .


%45
\cite{9978983} abordam um método chamado "PUTraceAD" Positive-Unlabeled ,para  detecção de anomalias em \textit{tracing} de microsserviços . A arquitetura proposta envolve três etapas: Embedding de Span, Construção de Grafo de Rastros e Treinamento do Modelo, utilizando uma GNN (Graph Neural Network) e aprendizado PU para detectar anomalias em trancing. O conjunto de dados usado para os experimentos é chamado de TrainTicket, um sistema de referência de microsserviços, e os experimentos foram realizados em um cluster Kubernetes. Os resultados dos experimentos avaliam a eficácia e eficiência do PUTraceAD, bem como o impacto de diferentes configurações ,

%46
Neste artigo, os autores\cite{9985105} discutem sobre o Sistema de Gerenciamento de Dependências ou em inglês, Dependency Management System (DMS), para gerenciar dependências de serviços em sistemas em nuvem. O DMS é uma plataforma de ponta a ponta que oferece suporte completo ao ciclo de vida para garantir a confiabilidade do serviço, incluindo implantação inicial, atualização , otimização arquitetural pró-ativa e mitigação reativa de falhas. Os dados usados nos testes do DMS incluem informações de dependência coletadas de trancing distribuído, arquivos de configuração, consultas do orquestrador de serviços e relatórios de dependência de implantação.


%47
Com base nas questões relacionadas à implementação da observabilidade em sistemas de informações hospitalares (HIS) os autores \cite{10004053} abordam uma arquitetura de microsserviços e AIOps (Inteligência Artificial para Operações de TI). De forma contextual, o artigo faz pesquisa literária e apresenta um resumo de conceitos relacionados, incluindo as definições de observabilidade e HIS, requisitos e ideias de solução de observabilidade em cenários específicos, além de responder a perguntas pré-propostas. A arquitetura proposta envolve a aplicação de microsserviços e AIOps para alcançar a observabilidade do HIS, sendo fatores-chave indicadores como qualidade e escala de dados. O documento também fornece sugestões para o departamento de TI do hospital sobre como lidar com essas questões. 

%48
\cite{10020986} abordam uma análise de AIOps (Inteligência Artificial para Operações de TI) na gestão unificada de resiliência de dados em data lakehouses. O artigo propõe soluções para prever violações de Recovery Point Objective  (RPO) e fornecer sugestões aos SREs (Engenheiros de Confiabilidade do Site) sobre como configurar recursos do sistema para evitar violações de RPO. O artigo utiliza aprendizado supervisionado em conjunto com análise de séries temporais e propõe um modelo de aprendizado de máquina com uso de métodos combinados de aprendizado online e offline e filtragem de solicitações previstas para filtrar solicitações futuras estáveis.


\subsection{Trabalhos Correlatos - ano de 2023}\label{trab_correlatos_23}


%56
\cite{10098585} os autores fornecem informações sobre a implementação de baselines usando Python e regressão linear como o modelo de aprendizado de máquina, as questões de pesquisa e o design experimental, a aplicação na análise de desempenho, distribuições de latência, o uso de cargas de trabalho continuamente variáveis e a configuração experimental.

%57
\cite{10113794} abordam o desenvolvimento de um sistema de detecção de outlier chamado OutSpot para datacentes com alto nivel de desempenho e alta criticidade, que basicamente atuam com o fornecimento de streaming de videos. O sistema tem como objetivo detectar outliers nos KPIs coletados desses datacenters.  O modelo de ML utilizado, integra Hierarchical Agglomerative Clustering (HAC), com Conditional Variational Autoencoder (CVAE). O HAC é aplicado para agrupar os KPIs com base em seus padrões, e as informações de agrupamento de cada KPI são, então incorporadas ao método CVAE. Essa abordagem permite que o OutSpot detecte outliers para KPIs em larga escala com padrões diversos A arquitetura proposta do OutSpot envolve dividir os conjuntos de dados coletados em conjuntos de treinamento e teste. O conjunto de treinamento consiste em dados dos primeiros 7 dias, enquanto o conjunto de teste é composto por dados do último dia. O conjunto de teste é rotulado por operadores experientes usando uma ferramenta desenvolvida pelos autores. Os outliers são rotulados por três operadores, sendo a decisão final tomada quando seus rótulos divergem.

\chapter{Fundamentação Teórica}
\label{cap_fundamentacao-teorica}

Nesta seção, será abordado os conceitos e teorias que formam a base deste estudo. O entendimento desses conceitos é fundamental para a compreensão das técnicas e metodologias empregadas neste trabalho. Primeiramente, será discutido o conceito de Inteligência Artificial (IA) e como essa tecnologia tem evoluído ao longo dos anos. Posteriormente, será explicado um subcampo específico da IA, conhecido como \textit{AIOps} (Inteligência Artificial para Operações de TI), que combina IA e análise de dados para melhorar e otimizar as operações de TI. O conhecimento adquirido sobre IA e AIOps será crucial para entender como essas tecnologias podem ser aplicadas em ambientes de TI para melhorar a detecção de anomalias e análise preditiva de séries temporais \cite{russell2016artificial, gardner2017artificial}.

\section{Transformer em Time Series}

Time Series Transformer (TST) é uma adaptação dos Transformer para lidar especificamente com séries temporais \cite{lim2019temporal}. Ele mantém o mecanismo de atenção dos Transformer, permitindo capturar dependências temporais em diferentes escalas de tempo.

Um dos aspectos críticos na utilização de TST é a escolha adequada dos hiperparâmetros, incluindo o número de camadas, a dimensão do modelo, e o número de cabeças de atenção. Esses hiperparâmetros podem ter um impacto significativo no desempenho do modelo e precisam ser ajustados de acordo com a especificidade dos dados e o problema em questão.

Quando aplicado a dados de séries temporais armazenados em Prometheus, o TST pode ser capaz de capturar padrões complexos e dependências temporais nas métricas de desempenho de sistemas de TI. Isso pode permitir a identificação mais precisa de problemas, bem como prever tendências futuras, o que é essencial para otimização e planejamento de recursos em ambientes de TI complexos.




\section{Inteligência Artificial}

A Inteligência Artificial (IA) é um campo de estudo que busca desenvolver máquinas que possam imitar ou simular o comportamento inteligente humano \cite{russell2016artificial}. Isso envolve a capacidade de aprender com experiências, compreender linguagem natural, reconhecer padrões, e tomar decisões de forma autônoma. 

A IA tem suas raízes em diversas disciplinas, incluindo ciência da computação, matemática, psicologia, neurociência, linguística e filosofia. O campo de IA tem evoluído rapidamente nos últimos anos, graças aos avanços em algoritmos, aumento na capacidade de processamento de computadores, e a disponibilidade de grandes volumes de dados \cite{poole2017artificial}.

Uma das subáreas mais proeminentes da IA é o aprendizado de máquina (Machine Learning), que foca em desenvolver algoritmos que possam melhorar seu desempenho e tomar decisões com base nos dados que recebem \cite{mitchell1997machine}. O aprendizado profundo (Deep Learning), que é um subcampo do aprendizado de máquina, tem se mostrado particularmente eficaz em tarefas como reconhecimento de imagem e processamento de linguagem natural, devido à sua capacidade de modelar relações complexas em grandes conjuntos de dados \cite{lecun2015deep}.

\section{AIOps}

AIOps, ou Inteligência Artificial para Operações de TI, é um termo que foi cunhado pelo Gartner e refere-se à aplicação de técnicas de IA em operações de TI \cite{gardner2017artificial}. O objetivo principal de AIOps é automatizar e melhorar aspectos das operações de TI, como monitoramento, gerenciamento, e análise de grandes volumes de dados gerados pelos sistemas e infraestruturas de TI.

AIOps combina big data e aprendizado de máquina para proporcionar análises mais inteligentes e automatizar tarefas que são tradicionalmente manuais e demoradas. Isso inclui detecção de anomalias, correlação de eventos, identificação de causa raiz, e análise preditiva \cite{padhy2018survey}. 

Ao utilizar AIOps, as equipes de operações de TI podem se tornar mais proativas, ao invés de reativas, na abordagem de problemas. Isso significa que, em vez de simplesmente responder a problemas depois que eles ocorrem, AIOps permite que as organizações prevejam problemas e os previnam antes que causem impacto \cite{sill2019aiops}. Em um cenário onde a complexidade dos sistemas de TI está aumentando e a tolerância a tempo de inatividade está diminuindo, AIOps tem o potencial de ser uma ferramenta inestimável para melhorar a eficiência e a confiabilidade das operações de TI.


\section{Time Series, Prometheus e Métricas}

Séries temporais, ou Time Series, são conjuntos de dados ordenados no tempo, frequentemente compostos por medições feitas em intervalos sequenciais \cite{shumway2017time}. Exemplos comuns de séries temporais incluem preços de ações, medições meteorológicas, e, no contexto desta pesquisa, métricas de desempenho de sistemas de TI.

No ambiente de TI, séries temporais são extremamente relevantes para o monitoramento e otimização de sistemas. A coleta e análise de métricas como uso de CPU, memória, tráfego de rede e latência, são essenciais para manter a estabilidade e a eficiência de sistemas complexos, particularmente em ambientes com virtualização, containers e Kubernetes.

Prometheus é um sistema de monitoramento e alerta de código aberto, criado em 2012 pela SoundCloud, e é atualmente mantido pela Cloud Native Computing Foundation \cite{brazil2019prometheus}. Ele foi projetado para confiabilidade e escalabilidade, sendo uma das soluções mais populares para coleta de métricas em ambientes de nuvem nativa, em parte devido à sua integração nativa com o Kubernetes. Prometheus permite coletar métricas de sistemas de TI em intervalos de tempo regulares e armazená-los em uma base de dados de séries temporais. Sua popularidade e confiabilidade fazem dele uma escolha adequada para o cenário proposto nesta dissertação.

\section{Deep Learning, RNN e Transformer}

Deep learning é um subcampo do aprendizado de máquina que envolve redes neurais com várias camadas, permitindo modelar relações complexas nos dados \cite{lecun2015deep}. Recurrent Neural Networks (RNN) são uma classe de redes neurais que são particularmente adequadas para lidar com séries temporais e sequências, pois são capazes de manter um estado interno que pode capturar informações sobre etapas anteriores na sequência \cite{elman1990finding}.

No entanto, as RNNs têm limitações, como a dificuldade de capturar dependências de longo alcance em sequências. Transformer foram introduzidos em um artigo seminal de Vaswani et al. em 2017 como uma alternativa às RNNs \cite{vaswani2017attention}. Eles utilizam mecanismos de atenção para ponderar diferentes partes de uma sequência de entrada, permitindo capturar dependências de longo alcance de forma mais eficaz.

Os Transformer mostraram ter um enorme potencial não apenas em processamento de linguagem natural, mas também em análise de séries temporais. Eles podem ser particularmente úteis em ambientes de TI complexos, onde a detecção de padrões sutis ao longo do tempo pode ser crítica para identificar problemas e otimizar o desempenho.


\chapter{Metodologia}

Esta pesquisa emprega uma abordagem baseada na arquitetura \textit{Transformer} para analisar séries temporais de métricas de desempenho de sistemas de TI coletadas pelo Prometheus. A arquitetura Transformer foi originalmente desenvolvida para processamento de linguagem natural \cite{vaswani2017attention}, mas foi posteriormente adaptada para lidar com séries temporais, demonstrando desempenho superior em várias tarefas, incluindo previsão e detecção de anomalias \cite{lim2019temporal}.

Os dados serão coletados usando Prometheus, que armazena métricas de séries temporais de diversos componentes de um sistema de TI, como uso de CPU, memória, e tráfego de rede. A escolha de utilizar o Prometheus é baseada na sua confiabilidade, eficiência e popularidade em ambientes de nuvem nativa \cite{brazil2019prometheus}.

A metodologia envolve o pré-processamento dos dados coletados para estruturá-los em um formato que possa ser alimentado em um modelo de Transformer. O modelo será então treinado para aprender representações de alto nível das métricas de desempenho e suas dependências temporais. A avaliação do modelo será realizada através de técnicas de validação cruzada e comparação com benchmarks estabelecidos.

\section{Datasets}

Neste trabalho será utilizado um dataset personalizado, coletado de um ambiente real de TI usando Prometheus. Utilizar dados reais oferece uma oportunidade única de avaliar o desempenho do modelo em um cenário prático, o que é crítico para entender a aplicabilidade e as limitações da abordagem proposta.

Este dataset inclui métricas de desempenho típicas, como uso de CPU, uso de memória, latência de rede, e tráfego de rede, ao longo do tempo. Estas métricas são comuns em estudos de séries temporais em ambientes de TI 
%\cite{calheiros2011workload}.

\begin{table}[h!]
    \centering
    \begin{tabular}{|c|c|}
        \hline
        \textbf{Feature} & \textbf{Descrição} \\
        \hline
        timestamp & Data e hora da medição \\
        \hline
        cpu\_usage & Percentual de uso da CPU \\
       

 \hline
        memory\_usage & Uso de memória em bytes \\
        \hline
        network\_latency & Latência de rede em milissegundos \\
        \hline
        network\_traffic & Tráfego de rede em bytes por segundo \\
        \hline
    \end{tabular}
    \caption{Features do dataset escolhido}
    \label{tab:dataset_features}
\end{table}


\section{Pre-processamento}

O pré-processamento de dados é um passo crítico para garantir que o modelo \textit{Transformer} possa extrair informações significativas das séries temporais coletadas pelo \textit{Prometheus}. Devido à natureza das séries temporais, alguns aspectos específicos precisam ser abordados.

Primeiramente, será necessário lidar com possíveis dados faltantes, pois as séries temporais podem ter lacunas%\cite{che2018recurrent}
. Será empregado uma estratégia comum, denominada a interpolação, para preencher os dados faltantes, utilizando métodos como interpolação linear ou preenchimento baseado em valores vizinhos.

Em seguida, as séries temporais precisam ser normalizadas. A normalização é essencial para que o modelo \textit{Transformer} treine de forma eficaz, uma vez que a arquitetura depende de atenção autorregressiva e, portanto, é sensível à escala dos dados \cite{vaswani2017attention}. Uma abordagem comumente utilizada é reescalar os dados para ter uma média de zero e um desvio padrão de um. Serão realizados testes utilizando esta técnica nesta etapa.

Além disso, será importante segmentar os dados em janelas temporais. Modelos de séries temporais, incluindo \textit{Transformer}, operam em segmentos de dados chamados janelas. A segmentação adequada é vital para capturar dependências temporais \cite{bai2018empirical}.

Finalmente, a divisão dos dados em conjuntos de treinamento, validação e teste será fundamental para avaliar o desempenho do modelo. Isso será realizado de forma a preservar a ordem temporal dos dados.

\section{Análise Preditiva com Transformer}

Para realizar análise preditiva nas séries temporais, será empregado o uso do modelo \textit{Transformer}. Inicialmente projetados para tarefas de processamento de linguagem natural, os Transformer demonstraram ser eficazes em modelar dependências temporais em séries temporais \cite{lim2019temporal}.

O modelo \textit{Transformer} baseia-se na atenção de auto-regressão, onde o modelo aprende a ponderar diferentes partes da entrada de acordo com sua relevância. Os hiperparâmetros, como o número de camadas de atenção, a dimensionalidade dos vetores de atenção, e o número de "cabeças de atenção", serão ajustados para otimizar o desempenho do modelo \cite{vaswani2017attention}.

Durante o treinamento, o modelo é alimentado com janelas de dados e possui como \textit{output} os  valores futuros previstos. Serão avaliados quais funções de perda possuem maior \textit{fit}  com o problema, alguns exemplos são: Root Mean Square Error (RMSE) e o Mean Absolute Error (MAE).

Neste contexto, é relevante comparar o \textit{RMSE} que atribui mais peso a erros grandes, enquanto o MAE trata todos os erros igualmente. Em cenários em que é importante que o modelo esteja mais próximo dos valores reais e não seja excessivamente influenciado por \textit{outliers}, o MAE poderá ser uma escolha mais adequada \cite{willmott2005advantages}. O MAE estimula o modelo a encontrar valores que são mais representativos da realidade e não tendem em direção à média.

Além disso, a otimização dos hiperparâmetros será realizada por meio de técnicas como \textit{grid search}, e, \textit{random search}, para encontrar a combinação que minimiza o \textit{MAE}  e RMSE no conjunto de validação. 
\chapter{Considerações Parciais} \label{cap:conclusion}

Através da observação das tendências extraídas de trabalhos correlatos e documentadas na Seção de Revisão Sistemática da Literatura (\ref{cap_revisao-sistematica}), aplicação de técnicas de \textit{machine learning} são cada vez mais utilizadas no contexto de operações de T.I, buscando não apenas automatizar tarefas, mas também identificar causa raiz de problemas complexos, prever tendência de consumo de recursos e identificar anomalias.

A utilização de uma arquitetura Transformer para análise preditiva em séries temporais com métricas de consumo dos dispositivos, mostra-se promissora, dado sua capacidade de capturar dependências de longo alcance e tendências complexas, diferenciando anomalias de sazonalidades, tornando-se em uma ferramenta muito relevante para AIOps, ajudando a melhorar a eficiência e a confiabilidade dos sistemas.


\section{Cronograma}
Apresenta-se a Tabela \ref{tab:cronograma} como cronograma até a defesa:

\begin{table}[h]
\begin{tabular}{|c|c|c|c|c|c|c|c|c|c|c|c|}
\hline
 & Ago & Set & Out & Nov & Dez & Jan & Fev & Mar & Abril & Maio & Jun \\ \hline
\begin{tabular}[c]{@{}c@{}}Correções \\ Pontuadas\\ pela Banca\end{tabular} & X & X &  &  &  &  &  &  &  &  &  \\ \hline
\begin{tabular}[c]{@{}c@{}}Metodologia / \\ Experimentos\end{tabular} & X & X & X &  &  &  &  &  &  &  &  \\ \hline
Desenvolvimento & X & X & X & X &  &  &  &  &  &  &  \\ \hline
Qualificação &  &  &  &  & X &  &  &  &  &  &  \\ \hline
\begin{tabular}[c]{@{}c@{}}Escrita e Submissão\\ de Artigos\end{tabular} &  &  &  &  &  & X & X & X & X & X &  \\ \hline
Defesa &  &  &  &  &  &  &  &  &  &  & X \\ \hline
\end{tabular}
\caption{Cronograma de Trabalho até a Defesa}
\label{tab:cronograma}
\end{table}

%\input{bkp_chapters/introducao.tex}

% ----------------------------------------------------------
% ELEMENTOS PÓS-TEXTUAIS
% ----------------------------------------------------------
\postextual
% ----------------------------------------------------------

% ----------------------------------------------------------
% Referências bibliográficas
% ----------------------------------------------------------
\pagestyle{empty}
\bibliography{chapters/references.bib}

%\addbibresource{chapters/references.bib}  % Specify the path to your bibliography file
% Your document content goes here
%\printbibliography  % Print the bibliography


\end{document}


\end{document}
% END ----------------------------------------------------------
